%%% À compiler *deux fois* en utilisant *lualatex* 
%%% La fonte Arial doit être installée sur le système
\documentclass{success_stories}

% Méta-données, ne sont pas affichées dans le document
\MetaData{
    author={Blanc, Xavier},
    date={2017-06-16},
    keywords={Big data, machine learning},
    version={working},
    lang={english},
    identifier={safety_line},
}


\begin{document}

% Titre, résumé en une ligne, secteur H2020, 
\Setup{
    name=Safety Line: big data applied to air transport,
    oneline=Risk estimation in air transport using machine learning,
    h2020={aeronautics, data, risk},
    sector=Security in transport
}

%%% Les challenges H2020
%    Health, demographic change and wellbeing;
%    Food security, sustainable agriculture and forestry, marine and maritime and inland water research, and the Bioeconomy;
%    Secure, clean and efficient energy;
%    Smart, green and integrated transport;
%    Climate action, environment, resource efficiency and raw materials;
%    Europe in a changing world - inclusive, innovative and reflective societies;
%    Secure societies - protecting freedom and security of Europe and its citizens.

% Description du partenaire académique
\SetupAcademic{
    name=LSTA,
    logo=UPMC_Sorbonne_Universites.png,
    description={LSTA stands for "Theoretical and Applied Statistics Laboratory". It is the statistics research department of Université Pierre et Marie Curie, in Paris.}
}

% Description du partenaire industriel
\SetupIndustrial{
    name=Safety Line,
    logo=NEW-logo-safetyline2.png,
    description = {The company Safety Line with created in 2010. It offers solutions innovative solutions for
      aviation safety and efficiency. These solutions are based on data analysis and machine learning.}
}

% Illustration
\SetupIllustration{
    img={aircraft.JPG},
    height=6cm
}

% Description mathématique (1ère page)
\Setup{math description={
Baptiste Gregorutti's PhD aimed at using machine learning techniques on flight data. The objective was to use a
bottom-up strategy, starting from the question of finding flights which were potentially dangerous, even though no
accident occured. 

A first analysis indicates that data are highly dependent, so a natural approach is to select relevant data for the
events under consideration. In order to do so, the method of random forests was applied with success. In a second
step, this method was extended to multivariate functionnal data, allowing for the use of more general criteria.
}}

% Description du problème et du contexte
\Setup{problem description={
    The data from airplane's flight recorders (black boxes) are stored by companies, but they remain unused, except
    when an incident occurs. The idea of the PhD thesis of Baptiste Gregorutti (supervised by Gérard
    Biau, Bertrand Michel and Philippe Saint-Pierre) was to use these data in order to detect risky situations, even for an eventless flight.  
    }}

% But de la collaboration
\Setup{goal={
Develop an algorithm allowing for:
\begin{enumerate}
\item the detection of risky flights;
\item finding variables which could be related to these risks.
\end{enumerate}
Propose an ready-to use implementation, with the possibility to tune the paremeters of the algorithm. 
    }}

% Résultat (2ème page)
\Setup{results={
    The method proved to be efficient in several relevant use-cases. After the completion of the PhD, a software
    was developped, using this algorithm: FlightScanner. This software is now available as a product of the
    company. Baptiste Gregorutti has been hired as a researcher in data science. In January 2017, he has become
    research manager.
}}


% Highlight et contribution
\Setup{
    highlight={Use of up-to-date \focus{machine learning} algorithms in a risk management software.
    },
    contribution={New product allowing for efficient data analysis of flight data.}
}

%%%%%%%%%%%%%%%%%%%%%%%%%%%%%%%%%%%%%%%%%%%%%%%%%%%%%%%%%%%%%%%%%%%%%%
%%%%%%%%%%%%%%%%%%%%%%%%%%%%%%%%%%%%%%%%%%%%%%%%%%%%%%%%%%%%%%%%%%%%%%
%%%%%%%%%%%%%%%%%%%%%%%%%%%%%%%%%%%%%%%%%%%%%%%%%%%%%%%%%%%%%%%%%%%%%%
\CreateStory

\end{document}




%%% Local Variables:
%%% mode: latex
%%% TeX-master: t
%%% End:
