%%% À compiler *deux fois* en utilisant *lualatex* 
%%% La fonte Arial doit être installée sur le système
\documentclass{success_stories}

% Méta-données, ne sont pas affichées dans le document
\MetaData{
    author={Doll\'e, Guillaume},
    date={2017-05-11},
    keywords={Modelisation,FEM},
}


\begin{document}

% Titre, résumé en une ligne, secteur H2020, 
\Setup{
    name=SIVIBIR++ Simulation of a bioreactor,
    oneline=Simulation of a virtual bioreactor landfill using Feel++ framework,
    h2020={Secure, clean and efficient energy},
    sector=ENERGY ENVIRONMENT
}

%%% Les challenges H2020
%    Health, demographic change and wellbeing;
%    Food security, sustainable agriculture and forestry, marine and maritime and inland water research, and the Bioeconomy;
%    Secure, clean and efficient energy;
%    Smart, green and integrated transport;
%    Climate action, environment, resource efficiency and raw materials;
%    Europe in a changing world - inclusive, innovative and reflective societies;
%    Secure societies - protecting freedom and security of Europe and its citizens.


% Description du partenaire académique
\Setup{academic/.cd,
    name=cemosis,
    logo=logoCemosis.pdf,
    description={Cemosis is hosted by the Institute of Advanced Mathematical
        Research (IRMA) and was created in January 2013. Cemosis mainly relies
        currently on the team Modeling and Control of the IRMA}
}

% Description du partenaire industriel
\Setup{industrial/.cd,
    name=Charier,
    logo=logoCharier.jpg,
    description = {
        The Charier company concentrates on five businesses: Aggregates, Earthworks, Roads and Urban Works, Civil Engineering and Special Works, Waste Recovery.
    }
}


% Illustration
\Setup{
    illustration/img={sivibirpp.png},
    illustration/height=6cm
}

% Description mathématique (1ère page)
\Setup{math description={
        A complex model involving seven equations to modelize the chemical
        reactions inside the bioreactor has been proposed. The model
        has been implemented using Feel++ a finite element library using
        Galerkin Methods. The model take into account the
        \begin{bullets}
        \item Carbon consumption by bacteria,
        \item Gaz mixture production,
        \item Water injections and evaporation,
        \item Temperature inside an alveolus.
        \end{bullets}
}}

% Description du problème et du contexte
\Setup{problem description={
    The principle of a bioreactor landfill consists in accelerating
    biodegradation of wastes. Maintaining an optimal degree of humidity, in
    particular fostering leachates recirculation helps to achieve this goal.
    The chemical reaction produces gaz which is reprocessed in another form of
    energy (heat, electricity, …).
    }}

% But de la collaboration
\Setup{goal={
        The goal of this project is to modelize the chemical reaction occuring
        inside a real bioreactor, then to generalize the whole bioreactor
        operating system. The final goal is to control the gaz production in
        order to optimize bioreactor yield in function of the energy needs.
%\begin{bullets}
%    \item Reduce the maximum and average waiting time while maintaining a high service quality at Front Office
%    \item Reduce of resource requirements even with increase in demand rate 
%    \item Improve assignments of Back Office and sales tasks
%\end{bullets}
    }}

% Résultat (2ème page)
\Setup{results={
        A multiphasic model has been established during a math-entreprise
        studying week “Semaine d’Étude Math-Entreprise” (SEME). A C++
        implementation of the model has been proposed during the “Centre d’Été
        Mathématique de Recherche Avancée en Calcul Scientifique” (CEMRACS
        2015) using the FEEL++ library which is developed in local.
}}


% Highlight et contribution
\Setup{
    highlight={
        Better biological understanding
        \focus{helps improving} alveolus construction,
        increasing \focus{production}
    },
    contribution={Helps to understand\\
        biological reaction inside the bioreactor
	}
}

%%%%%%%%%%%%%%%%%%%%%%%%%%%%%%%%%%%%%%%%%%%%%%%%%%%%%%%%%%%%%%%%%%%%%%
%%%%%%%%%%%%%%%%%%%%%%%%%%%%%%%%%%%%%%%%%%%%%%%%%%%%%%%%%%%%%%%%%%%%%%
%%%%%%%%%%%%%%%%%%%%%%%%%%%%%%%%%%%%%%%%%%%%%%%%%%%%%%%%%%%%%%%%%%%%%%
\CreateStory





\end{document}



