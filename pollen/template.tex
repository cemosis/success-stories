%%% À compiler *deux fois* en utilisant *lualatex* 
%%% La fonte Arial doit être installée sur le système
\documentclass{success_stories}
\usepackage{mathtools}


% Méta-données, ne sont pas affichées dans le document
\MetaData{
    author={Fontanges, Richard},
    date={2017-08-04},
    keywords={},
    lang={english},
    version={working},
    identifier={pollen},
}


\begin{document}

% Titre, résumé en une ligne, secteur H2020, 
\Setup{
    name=Pollen Metrology : Data fusion for Metrology,
    oneline= Control and measurements of nanomaterials\\ by aggregation and analysis of different sources of images,
    h2020={Health, Secure Energy}, % ??? 
    sector={Electronics, Materials}
}

% Description du partenaire académique
\SetupAcademic{
    name=Laboratoire Jean Kuntzmann / Université Grenoble-Alpes,
    logo={LJK_UGA.png},
    description={}
}

% Description du partenaire industriel
\SetupIndustrial{
    name=Pollen Metrology,
    logo=Pollen.png,
    description = {
POLLEN Technology is a startup founded on September 29
Provides turnkey solutions to industrial and research laboratories in the field of nanometrology}
}


%Illustration
\SetupIllustration{
    img={illustration_pollen.png},
   height=5cm
}

% Description mathématique (1ère page)
\Setup{math description={
From a selection of  images types (Scanning Electron Microscopy, Transmission Electron Microscopy, Atomic Force Microscopy), appropriate computational methods are chosen.
As a preliminary step for classification, regions of interest are detected using SVMs technics (Support Vector Machines) and SIFT descriptors (Scale Invariant Feature Transform).
The next step, analysis of AM-FM images of nanoparticules, has been conducted in partnership with LNE (National Laboratory of Metrology and Testing) and was the object of a PhD thesis. 
Finaly, data fusion is applied  using a statistical approach with aggregation of estimators to obtain the wanted characteristics of the material. 
  }}

% Description du problème et du contexte
\Setup{problem description={
By the 2000s, nanomaterial became industrial products, thus nano-processes need to be (nano)mastered.
This means working on high density images / huge amount of data in an efficient way, i.e automated reliable, consistent and fast analysis.
    }}

% But de la collaboration
\Setup{goal={
	\begin{bullets}
	\item Getting workable images
	\item Image analysing
	\item Data fusionning to extract the best results
	\end{bullets}
The process has to be fully automated and configurable for the end-user (client).
    }}

% Résultat (2ème page)
  \Setup{results={
As a result, robust algorithms for preprocessing (removing of the trend, segmentation and detection of the nano-objects) have been selected and adapted. Moreover, data fusion methods have been added in a innovative way (for this sector of industry)
\begin{bullets}
\item Automated particle analysis	99.9\,\% success on LNE data 	
\item Automated AFM flattening 	100\,\% success
\item Automated Fusion: Combination of two estimators coming from multiple measurement techniques has been developed 
\end{bullets}
}}


% Highlight et contribution
\Setup{
  highlight={
      for  the \underline{end user}:\\
\textbullet Higher Production Yield\\
\textbullet Automated results (vs~manually)
from Week to Minutes time analysis !

\medskip

for \underline{Pollen}:\\
\textbullet Taking over of market share\\
\textbullet Recognised as innovative
          },
    contribution={Accurate 3D process control}
}

%%%%%%%%%%%%%%%%%%%%%%%%%%%%%%%%%%%%%%%%%%%%%%%%%%%%%%%%%%%%%%%%%%%%%%
%%%%%%%%%%%%%%%%%%%%%%%%%%%%%%%%%%%%%%%%%%%%%%%%%%%%%%%%%%%%%%%%%%%%%%
%%%%%%%%%%%%%%%%%%%%%%%%%%%%%%%%%%%%%%%%%%%%%%%%%%%%%%%%%%%%%%%%%%%%%%
\CreateStory





\end{document}



