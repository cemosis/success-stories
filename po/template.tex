%%% À compiler *deux fois* en utilisant *lualatex* 
%%% La fonte Arial doit être installée sur le système
\documentclass{success_stories}
\usepackage{mathtools}


% Méta-données, ne sont pas affichées dans le document
\MetaData{
    author={Hild, Romain},
    date={2017-05-10},
    keywords={Navier-Stokes, curl},
}


\begin{document}

% Titre, résumé en une ligne, secteur H2020, 
\Setup{
    name=$\mathrm{NS^2_{++}}$: Non Standard Navier-Stokes in Feel++,
    oneline=A non standard strategy for solving the incompressible Navier-Stokes equations,
    h2020={Smart, green and integrated transport},
    sector=MECHANICS AND MECHATRONICS
}

% Description du partenaire académique
\Setup{academic/.cd,
    name=Cemosis,
    logo=Cemosis_logo.png,
    description={Plastic Omnium is the world’s leading supplier of plastic and composite painted exterior components and modules. Its Auto Exterior Division supports our customers worldwide with 87 plants in 18 countries.}
}

% Description du partenaire industriel
\Setup{industrial/.cd,
    name=Plastic Omnium,
    logo=po.jpg,
    description = {Innovation through modeling and simulation.\\
Cemosis is hosted by the Institute of Advanced Mathematical Research (IRMA) and was created in January 2013. Cemosis mainly relies currently on the team Modeling and Control of the IRMA.}
}

%Illustration
\Setup{illustration/img={mode45.png},
   illustration/height=5cm
}

% Description mathématique (1ère page)
\Setup{math description={
    The aim is to study a formulation of the Navier-Stokes equations involving the eigenfunctions of the curl operator.
    This formulation would decompose the solution in space and time.
    Thus, most of the computation is done once for the geometry, and each time step is not computationnaly intensive.
    This is done mainly in three steps:
    \begin{bullets}
    \item Separate the boundary conditions of the main problem,
    \item Seek the eigen modes of the rotational operator reliably and efficiently,
    \item Project the NS equation on the space spaned by these eigen modes to reduce the computational cost.
    \end{bullets}
  }}

% Description du problème et du contexte
\Setup{problem description={
    PlasticOmnium Automotive is a leading company of parts and plastic body modules.
    In order to keep his place and ahead of the European regulations concerning the emission of $\mathrm{CO_2}$, it must test the air flow over cars.
    For this, the company needs performing models of turbulence geometry for several million cells.
    }}

% But de la collaboration
\Setup{goal={
\begin{bullets}
\item Reduce the time and the computing power needed for the simulation
\item Increase the length of the simulation by a factor 10
\item Reduce the fuel consumption by a better aerodynamic
\end{bullets}
    }}

% Résultat (2ème page)
  \Setup{results={
      The resolution of the non standard formulation of the Navier-Stokes equation
      had to face several issues. We managed to solve efficiently and accurately the
      eigen problem of the curl operator and to project the solution on the space spanned
      by the eigen vectors. The last obstacle is the lifting of the boundary conditions.
      In order to fix this, the collaboration between Cemosis and Plastic Omnium
      continues and grows on the problematic of large scale post-processing of
      computational fluid dynamics simulations.
}}


% Highlight et contribution
\Setup{
  highlight={
    Deal with key\\
    component of the \focus{non standard}
    Navier-Stokes equation and
    start a \focus{partnership} between
    Cemosis and Plastic Omnium
    },
    contribution={Research on key\\
	Future Aspect}
}

%%%%%%%%%%%%%%%%%%%%%%%%%%%%%%%%%%%%%%%%%%%%%%%%%%%%%%%%%%%%%%%%%%%%%%
%%%%%%%%%%%%%%%%%%%%%%%%%%%%%%%%%%%%%%%%%%%%%%%%%%%%%%%%%%%%%%%%%%%%%%
%%%%%%%%%%%%%%%%%%%%%%%%%%%%%%%%%%%%%%%%%%%%%%%%%%%%%%%%%%%%%%%%%%%%%%
\CreateStory





\end{document}



