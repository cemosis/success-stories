%%% À compiler *deux fois* en utilisant *lualatex* 
%%% La fonte Arial doit être installée sur le système
\documentclass{success_stories}

% Méta-données, ne sont pas affichées dans le document
\MetaData{
    author={Dollé, Guillaume},
    date={2017-05-11},
    keywords={Optimization},
    lang={english}
}


\begin{document}

% Titre, résumé en une ligne, secteur H2020, 
\Setup{
    name=M4SE Math for Smart Energy,
    oneline=Optimization of energy storage within households,
    h2020={Secure, clean and efficient energy},
    sector=ENERGY ENVIRONMENT
}

%%% Les challenges H2020
%    Health, demographic change and wellbeing;
%    Food security, sustainable agriculture and forestry, marine and maritime and inland water research, and the Bioeconomy;
%    Secure, clean and efficient energy;
%    Smart, green and integrated transport;
%    Climate action, environment, resource efficiency and raw materials;
%    Europe in a changing world - inclusive, innovative and reflective societies;
%    Secure societies - protecting freedom and security of Europe and its citizens.

% Description du partenaire académique
\Setup{academic/.cd,
    name=cemosis,
    logo=logoCemosis.pdf,
    description={Cemosis is hosted by the Institute of Advanced Mathematical
        Research (IRMA, Strasbourg) and was created in January 2013. Cemosis mainly relies
        currently on the team Modeling and Control of the IRMA}
}

% Description du partenaire industriel
\Setup{industrial/.cd,
    name=Hager,
    logo=logoHager.jpg,
    description = {
        Manufacturer of electrical equipment for modular electrical
        installation, energy management, alarm and home and building automation
        system
    }
}

% Illustration
\Setup{illustration/img={hager_scheme.png},
    illustration/height=6cm
}

% Description mathématique (1ère page)
\Setup{math description={
        This project involves an optimization problems with constraints with some knowledges
        (predictive data).
        \begin{bullets}
        \item Minimize a cost function to fit the eroadmap in function of real energy production.
        \item Introduce constraints in the problem (Battery types, household production law imposed by governments).
        \item Add predictive data (weather prediction, house energy knowledges,\ldots) in the model to handle free schedule.
        \end{bullets}
}}

% Description du problème et du contexte
\Setup{problem description={
        With the multiplications of photovoltaic panels, wind turbines and the
        possibility to own its energy production, new problems emerged:
        \textbullet~Massive energy rejection causes electrical network disturbances.
        \textbullet~Energy distributors imposed regulates household consumption in function of stock exchange prices. 
        \textbullet~Energy waste due to non-consumed or non-selled energy during free schedule.
    }}

% But de la collaboration
\Setup{goal={
        The goal is first to control the energy
        storage via energy buffers (batteries) and the rejection in the
        electrical network with respect to distributors consigns (eroadmap).
        Minimizing energy costs, maximizing energy resale during free schedule.
    }}

% Résultat (2ème page)
\Setup{results={
A prototype of a working algorithm has been established during a 
6 month work financed by AMIES PEPS contracts and master internship. The solution
proposed is a strategy to fit the eroadmap taking into account predictive data.
This works solved a blocking problem for the company which patented the algorithm
and implemented it in the EMG controller for production.
}}


% Highlight et contribution
\Setup{
    highlight={
	Energy storage to\\[\smallskipamount]
	\focus{reduce} energy wastes,\\[\smallskipamount]
    \focus{guaranty} safe electrical network for distributors,\\[\smallskipamount]
    \focus{control} energy resale for households
    },
    contribution={Reduce energy\\
        waste, control comsumption,
        optimize energy resale
	}
}

%%%%%%%%%%%%%%%%%%%%%%%%%%%%%%%%%%%%%%%%%%%%%%%%%%%%%%%%%%%%%%%%%%%%%%
%%%%%%%%%%%%%%%%%%%%%%%%%%%%%%%%%%%%%%%%%%%%%%%%%%%%%%%%%%%%%%%%%%%%%%
%%%%%%%%%%%%%%%%%%%%%%%%%%%%%%%%%%%%%%%%%%%%%%%%%%%%%%%%%%%%%%%%%%%%%%
\CreateStory




\end{document}



