%%% SUCCESS STORIES 
%%%
%%% AMIES and FMSP, based on the Eu-Maths-In initiative
%%%
%%%----------------------------------------------------------------------------
%%% TEMPLATE EXAMPLE
%%% 
%%% The class file success stories as well as the proper fonts shall be installed in the appropriate directories.
%%%
%%% ILLUSTRATIONS 
%%%	Each sheet is illustrated with 
%%%	- one picture in the front page
%%%     - one academic logo (multiple logos must be merged into a single file)
%%%     - one industriel logo (multiple logos must be merged into a single file)
%%%	Various formats can be used for the pictures: .pdf, .png, .jpg. 
%%%	Please avoid heavy (≥ 1Mb) images.
%%% 
%%% Logo positionning may be tricky as there are various size. It will be fine-tuned during the production phase
%%%
%%% IDENTIFIER
%%%	Each sheet is identified through a single identifier, for example the compact name of the project. 
%%%	The file is named after this identifier. It is given in the MetaData section.
%%%
%%% LATEX 
%%%	The content should be typesetted using minimal LaTex features (only \emph{}, \textbf{}, \textit{}). 
%%%	Use the `bullet` environment of lists.
%%%
%%% Template version: 1.0 - 2020-01-02
%%% Author: Antoine Lejay
%%%----------------------------------------------------------------------------
\documentclass{../latex/success_stories}


% METADATA
%   At the exception of digital_twin and url, these metadata do not appear in the document but are useful to categorize the project.
\MetaData{
    author={Gouzé, Jean-Luc},  % name of the author
    contact={Busé, Laurent}, % name of the contact at AMIES
    date={2018-02-03}, % creation date 
    keywords={health}, % list of keywords
    lang={english},  % french ou english
    version={draft}, % draft or final version
    versionno={1}, % version number 
    identifier={exactcure}, % identifier
    digital_twin={true}, % if true, then the Digital Twin logo will be added on the second page. 
    url = {https://www.exactcure.com/}, % url to a page describing the project, or the society. The 'Info box' logo will be linked with this URL
}


\begin{document}

% TITLE and ABSTRACT
%   Title, one line abstract
%
%   The h2020 sector shall be taken among
%	-Agriculture and Fishing
%	-Information and Communication Technology
%	-Biomedicine and Health Care
%	-Logistics and Transport
%	-Construction
%	-Materials
%	-Chemical and Pharmaceutical Industry
%	-Mechanics and Mechatronics
%	-Economy and Finance
%	-Public Administration and Defense
%	-Electronics
%	-Service Management
%	-Energy and Environment
%	-Textiles, Clothing and Footwear
%	-Food
%	-Other (specify)
%
%   The industrial sector shall be taken among 
%	-Agriculture and Fishing
%	-Information and Communication Technology
%	-Biomedicine and Health Care
%	-Logistics and Transport
%	-Construction
%	-Materials
%	-Chemical and Pharmaceutical Industry
%	-Mechanics and Mechatronics
%	-Economy and Finance
%	-Public Administration and Defense
%	-Electronics
%	-Service Management
%	-Energy and Environment
%	-Textiles, Clothing and Footwear
%	-Food
%	-Other (specify)
\Setup{
    name={ExactCure: Your Medical Digital Twin},
    oneline= Calibration of pharmacokinetic models of drugs and integration of patient data,
    h2020={Health, demographic change and wellbeing}, 
    sector={ Biomedicine and Health Care, Chemical and Pharmaceutical Industry }
}

% ACADEMIC
%   Description of the academic partner
\SetupAcademic{
    name={BIOCORE, Inria Nancy Méditerranée, France},
    logo={inr_logo_eng_rouge.png},
    description={BIOCORE is an Inria team at Inria Sophia Antipolis (Nice, France) with an expertise in dynamical systems and control theory, especially in the field of biomathematics.}
}


% INDUSTRIAL
%   Description of the industrial partner
\SetupIndustrial{
    name={ExactCure, Nice, France},
    logo= {logo-exact-cure.png},
    description = {ExactCure (Nice, France) is a eHealth company that leverages a proprietary Artificial Intelligence dedicated to personalized medicine.},
}


% Mathematical description (front page)
\Setup{
    math description={
ExactCure leverages a proprietary Artificial Intelligence to create personalized biomodels of drugs by adjusting an ideal posology and tightly monitoring the therapy. 

Pharmacokinetic models reflect the fate of a drug in an organism, from
absorption to distribution, metabolism and elimination. They are systems of
Ordinary Differential Equations. Their dynamics is calibrated on populations of
patients thanks to clinical trials. Their parameters are random variables that
integrate the inter-patient variability. They are Non Linear Mixed Effect
Models.  Our ExactCure/Inria collaboration leverages skills in dynamical
systems and control theory in the context of pharmacokinetics modelling.
  }}

% Problem and context description (front page)
\Setup{
    problem description={
Inappropriate medications kill more people than car accidents do! 3 times more, 
costing France 10 billion euros of health insurance per year. 
Why is it so complex? We are all different, we differently respond to drugs.
    }}
  
% Goal of the collaboration (front page)
\Setup{goal={
	\begin{bullets}
	\item Analysis of the dynamic properties of pharmacokinetic structures
	\item Integration of lacunar patient data
	\item Mapping of population models to individual characteristics
	\end{bullets}
    }
}

% Illustration (front page)
\SetupIllustration{
    img={image-exact-cure.pdf},
    height=6cm,  % Height of the image.
   legend={Interface for your digital twin},
}

% Results (back page)
  \Setup{
      results={
In pharmacokinetics modelling, personalization means to estime/adjust kinetic parameters from personnalized data. 
We have developed cutting-edge technologies for rigorous and smart integration of such personalized data in our proprietary calibration pipeline. The underlying mathematical techniques form the cornerstone of very challenging collaborations between ExactCure and Inria.   
}}



% Highlights and contribution (back page)
\Setup{
  highlight={
      A significant advancement in our core\\ calculation algorithms which 
      opens the way to a best-in-class
      \Impact{personalization} algorithm for
      patients.
  },
    contribution={Personalized\\ Biomodeling of Drugs}
}

% Here we are!
\CreateStory

\end{document}
%%%
%%% END OF THE TEMPLATE FILE
